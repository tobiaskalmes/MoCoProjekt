\documentclass[	12pt,										%fontsize 12pt
								twoside,								%singlesided
								a4paper,								%Papierformat A4
								headsepline,						%seperate header with line
								DIV15,
								BCOR2cm,
								titlepage,							%use titlepage			
								parskip=half,
								cleardoublepage=empty,
								%listof=flat,
								toc=listof, 						%remove warning
								%intoc, 								%remove warning
								%liststotoc,
]{scrbook}
\usepackage{datatool}
%\usepackage{txfonts} %Schriftart Times New Roman
\usepackage[numbib]{tocbibind}
\usepackage{graphicx}%load before hyperref %remove warning
\usepackage[		
				pdfborder={0 0 0},		%print links without borders	
				pdftitle=Seminar,
				pdfsubject={Seminar},
				pdfauthor={Tobias Kalmes},
				linktocpage=true,		%seitenzahlen im toc als links darstellen
				colorlinks=false,
				linkcolor=blue,
				anchorcolor=blue, 		%Color for anchor text
				citecolor=blue,			%Color for bibliographical citations in text.
				filecolor=blue,			%Color for URLs which open local files.
				menucolor=blue,			%Color for Acrobat menu items.
				runcolor=blue,			%Color for run links (launch annotations).
				urlcolor=blue,			%Color for linked URLs. 	
				bookmarksopen=true,
				breaklinks=true,
				hyperfootnotes=false %remove warning
				]{hyperref}
\usepackage[thmmarks,hyperref]{ntheorem}
\usepackage[ngerman]{babel}
\usepackage[utf8]{inputenc}
\usepackage{csquotes}
\usepackage[T1]{fontenc}
\usepackage[right]{eurosym}
\usepackage{array}
\usepackage{graphicx}
\usepackage{caption}
\DeclareCaptionFont{white}{\color{white}}
\DeclareCaptionFormat{listing}{\colorbox{gray}{\parbox{\textwidth}{#1#2#3}}}
\captionsetup[lstlisting]{format=listing, labelfont=white, textfont=white}
\usepackage[
					backend=biber,
					style=alphabetic,
					natbib=true,
%					citestyle=authoryear-icomp,
					autocite=footnote
					]{biblatex}

\usepackage{scrhack} %remove warning

\usepackage{prettyref}
%\usepackage{titleref}
\usepackage{listings}
\usepackage{rotating}
\usepackage{ae}
\usepackage{times}
\usepackage[nonumberlist, acronym, toc, section]{glossaries}
\usepackage{setspace}
\usepackage{tocloft}
\usepackage[usenames,dvipsnames]{color}
\usepackage{wasysym}
\usepackage{array}
\usepackage{tabularx}
\usepackage{pdflscape}
\usepackage[table]{xcolor}
\usepackage{float}
\usepackage{ulem}



\usepackage{textcomp}

%prepare stuff
\restylefloat{figure}
\restylefloat{table}
%\setcounter{tocdepth}{5} 

\definecolor{mediumslateblue}{rgb}{0.48,0.41,0.93}
\definecolor{crimson}{rgb}{0.86,0.0,0.24}
\definecolor{darkgreen}{rgb}{0.196,0.8,0.196}%{50,205,50}
\definecolor{mediumorchid}{rgb}{0.73,0.08,0.827}

%define listing style
\lstset
{
	basicstyle = \setstretch{1}\footnotesize\ttfamily\color{black}, %Standardschrift
	numbers = left,
	numberstyle = \tiny,
	%stepnumber = 2, %Abstand zwischen Zeilennummern
	numbersep = 5pt, %Abstand zwischen Nummern und Text
	tabsize = 2,
	extendedchars = true,
	breaklines = true,
	breakatwhitespace = true,
	keywordstyle = \color{blue},
	frame = b,
	stringstyle = \color{black}, %Farbe der Strings
	showspaces = false,
	showtabs = false,
	xleftmargin = 17pt,
	framexleftmargin = 5pt,
	framexrightmargin = 4pt,
	framexbottommargin = 4pt,
	%backgroundcolor = \color{lightgray},
	showstringspaces = false,
	%commentstyle = \itshape\color{darkgreen}
}

%define languages for listings
\lstdefinelanguage{XML}
{
	%morecomment = [s]{<?}{?>},
	morecomment = [s]{<!--}{-->},
	morestring = [b]",
	%morestring = [s]{>}{<},
	identifierstyle = \color{mediumslateblue},
	keywordstyle = \color{crimson},
	stringstyle = \itshape\color{mediumorchid},
	commentstyle = \itshape\color{darkgreen},
	morekeywords = {version, encoding, type, startswith, autocommit, key, name, condition}
}

\lstdefinelanguage{OWL}
{
	%morecomment = [s]{<!--}{-->},
	morestring = [b]",
	%morestring = [s]{>}{<},
	identifierstyle = \color{mediumslateblue},
	keywordstyle = \color{crimson},
	stringstyle = \itshape\color{mediumorchid},
	commentstyle = \itshape\color{darkgreen},
	morekeywords = {and, or, not, only, some}
}

\lstloadlanguages{Java,XML,C++}
\bibliography{literature/literature}
\theoremlisttype{all}
%\makeglossary
%Footnotes on first appearance
%\defglsdisplayfirst[main]{#1#4\protect\footnote{#2}}
\makeglossaries
